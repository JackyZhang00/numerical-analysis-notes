\chapter{复习题}

注: 复习题大部分出自课后练习题, 但对其进行了修改, 使其所覆盖的知识范围更广. 在完成复习题时, 请注意以下几点:

\begin{enumerate}
    \item 这道题所覆盖的知识点, 注意其知识点, 而不仅仅是一道题本身;
    \item 这道题所要求的计算方法, 需按照指定方法解答(大多数问题可能有多种解答方式, 在命制题目时, 为达到"一题多练"的效果, 对原题中部分算法进行了扩展)
    \item 本复习题不配套答案, 大多数题目的正确性可自行检验(理论分析, 程序计算等)
\end{enumerate}

部分题目可能进行了修改, 但没有考虑到其正确性. 因此, 若有任何题目上的问题, 可通过相关方式联系.

\section*{绪论}

1. 对于
\begin{equation*}
    x_1^*=1.1021, x_2^*=0.031, x_3^*=385.6
\end{equation*}
其都为四舍五入得到的近似数, 求

1.1. 它们的有效数字

1.2. 它们的误差限与相对误差限

1.3. $x_1^*x_2^*x_3^*$的绝对误差限与相对误差限

2. 对于一个球体, 其半径为$R$

2.1. 若$R$的相对误差为2\%, 求其球体体积的相对误差

2.2. 若要求球体积的相对误差限为1\%, 则半径$R$所允许的相对误差限为多少

3. 对多项式
\begin{equation*}
    f(x)=1+x+2x^2+3x^3+4x^4
\end{equation*}
使用秦九韶算法, 求$f(2)$的值.

\section*{插值法}
1. 给出$f(x)=\ln x$的数值表

\begin{table}[h!]
    \begin{tabular}{c|ccccc}
        \hline
        $x$&0.4&0.5&0.6&0.7&0.8\\
        \hline
        $\ln x$&-0.916291&-0.693147&-0.510826&-0.357765&-0.223144\\
        \hline
    \end{tabular}
\end{table}

分别使用Lagrange插值法和Newton插值法, 构造线性插值和二次插值, 计算$\ln0.54$的近似值. 并估计其误差

2. 给定下面数值表
\begin{table}[h!]
    \begin{tabular}{c|ccccc}
        \hline
        $x$&1&-1&2\\
        \hline
        $f(x)$&0&3&4\\
        \hline
    \end{tabular}
\end{table}

2.1. 使用Newton插值法, 构造满足上述条件的插值多项式($P_2(x)$)

2.2. 在满足上面数值表的前提下, 若同时满足$f'(1)=1$, 试通过构造重节点差商表, 构造插值多项式($P_4(x)$)

\section*{函数逼近与拟合}

1. 对于函数$f(x)=e^x$, 

1.1. 求其在$[-1,1]$上最佳一次逼近多项式, 并估计其误差

1.2. 求其在$[-1,1]$上在$\spn{1,x}$上的最佳平方逼近多项式, 并估计其误差

1.3. 使用Legendre多项式做正交多项式, 求其在$[-1,1]$上的最佳平方逼近多项式, 并估计其误差

2. 设$f(x)=x^4+3x^3-1$, 通过Chebyshev多项式, 在$[0,1]$区间构造其三次最佳逼近多项式, 并估计其误差限.

3. 对于下面的数据

\begin{table}[h!]
    \begin{tabular}{c|ccccc}
        \hline
        $x_i$&19&25&31&38&44\\
        \hline
        $y_i$&19.0&32.3&49.0&73.3&97.8\\
        \hline
    \end{tabular}
\end{table}

使用最小二乘法, 完成下面的拟合

3.1. 使用$y=ax$的形式

3.2. 使用$y=a+bx^2$的形式

3.3. 使用$y=a+bx+cx^2$的形式

\section*{数值积分}

1. 对于下面的求积公式, 确定其待定系数, 使其代数精度尽可能高, 并判断其代数精度

1.1. \begin{equation*}
    \int_{-2h}^{2h}f(x)\dd{x}\approx A_{-1}f(-h)+A_0f(x)+A_1f(h)
\end{equation*}

1.2. \begin{equation*}
    \int_{-1}^1f(x)\dd{x}\approx\left[f(-1)+2f(x_1)+3f(x_2)\right]/3
\end{equation*}

2. 使用下面的方法, 求解积分
\begin{equation*}
    \int_0^1\sqrt{x}e^x\dd{x}
\end{equation*}
并判断它们的误差

2.1. 梯形公式

2.2. Simpson公式

2.3. 取$n=4$的复化梯形公式与复化Simpson公式

2.4. Romberg算法

2.5. 正交多项式法的Gauss型求积公式

2.6. Gauss-Legendre求积公式

\section*{矩阵分析基础}

设矩阵
\begin{equation*}
    A=\mqty(
        6&5\\
        3&1
    )
\end{equation*}

1. 计算$A$行范数, 列范数, 2范数和F范数

2. 计算矩阵$A$的谱半径

3. 计算矩阵$A$的$\infty$-条件数和谱条件数

\section*{线性方程组的直接解法}

1. 用Gauss-Jordan方法求矩阵$A$的逆. 其中
\begin{equation*}
    A=\mqty(2&1&-3&-1\\
    3&1&0&7\\
    -1&2&4&-2\\
    1&0&-1&5)
\end{equation*}

2. 对线性方程组
\begin{equation*}
    \begin{cases}
        2x_1-x_2+x_3=4\\
        -x_1-2x_2+3x_3=5\\
        x_1+3x_2+x_3=6
    \end{cases}
\end{equation*}

使用下列方法求解方程组:

2.1. Gauss消去法

2.2. 列主元Gauss消去法

2.3. LU三角分解法

2.4. 紧凑格式列主元Doolittle三角分解法

2.5. 改进平方根法

\section*{线性方程组的迭代解法}

设方程组
\begin{equation*}
    \begin{cases}
        x_1-\frac{1}{4}x_3-\frac{1}{4}x_4=\frac{1}{2}\\
        x_2-\frac{1}{4}x_3-\frac{1}{4}x_4=\frac{1}{2}\\
        -\frac{1}{4}x_1-\frac{1}{4}x_2+x_3=\frac{1}{2}\\
        -\frac{1}{4}x_1-\frac{1}{4}x_2+x_4=\frac{1}{2}
    \end{cases}
\end{equation*}

求解该方程组的Jacobi迭代矩阵和Gauss-Seidel迭代矩阵, 并判断其收敛性.

\section*{非线性方程组求根}

1. 对于方程$f(x)=x^3-3x-1=0$

1.1. 使用二分法求$x_0=2$附近的根, 且要求误差小于0.05

1.2. 构造至少两种简单迭代格式, 并判断其收敛性

1.3. 使用Newton法解1.1, 且判断其收敛性和收敛速度.

2. 对于方程$x^2-x-1=0$, 若要求其误差小于0.05, 至少需要二分几次?

\section*{常微分方程数值解法}

1. 对于初值问题
\begin{equation*}
    \begin{cases}
        y'=x+y, &0<x<1\\
        y(0)=1
    \end{cases}
\end{equation*}, 取步长$h=0.1$

1.1. 使用Euler法构造其迭代格式, 并判断其局部截断误差

1.2. 使用改进的Euler法构造迭代格式, 并判断其局部截断误差

1.3. 若取$h=0.2$, 使用经典四阶Runge-Kutta方法求解上述初值问题, 并判断局部截断误差