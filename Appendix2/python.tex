\chapter{常见Runge-Kutta公式}

正如前文所说, Runge-Kutta公式具有无穷多种格式, 这里给出一些常见的RUnge-Kutta公式

\section{二阶Runge-Kutta方法}

\begin{equation*}
    \begin{cases}
        y_{i+1}=y_i+h[\lambda_1K_1+\lambda_2K_2]\\
        K_1=f(x_i,y_i)\\
        K_2=f(x_i+ph,y_i+phK_1)
    \end{cases}
\end{equation*}
其中
\begin{equation*}
    \lambda_1+\lambda_2=1, \lambda_2p=\frac{1}{2}
\end{equation*}

\subsection{二阶中点方法}
取$\lambda_1=0,\lambda_2=1,p=1/2$
\begin{equation*}
    \begin{cases}
        y_{i+1}=y_i+hk_2\\
        k_1=f(x_i,y_i)\\
        k_2=f(x_i+\frac{h}{2},y_i+\frac{h}{2}k_1)
    \end{cases}
\end{equation*}

\subsection{二阶Hune方法}
取$\lambda_1=1/4,\lambda_2=3/4.p=2/3$
\begin{equation*}
    \begin{cases}
        y_{i+1}=y_i+h\left(\frac{1}{4}k_1+\frac{3}{4}k_2\right)\\
        k_1=f(x_i,y_i)\\
        k_2=f(x_1+\frac{2h}{3},y_i+\frac{2h}{3}k_1)
    \end{cases}
\end{equation*}

\section{三阶Runge-Kutta方法}

\begin{equation*}
    \begin{cases}
        y_{i+1}=y_i+h(\lambda_1k_1+\lambda_2k_2+\lambda_3k_3)\\
        k_1=f(x_i,y_i)\\
        k_2=f(x_i+c_2h,y_i+c_2hk_1)\\
        k_3=f(x_i+c_3h,y_i+a_{31}hk_1+a_{32}hk_2)\\
        c_3=a_{31}+a_{32}
    \end{cases}
\end{equation*}
其中
\begin{equation*}
    \begin{cases}
        \lambda_1+\lambda_2+\lambda_3=1\\
        \lambda_2c_2+\lambda_3c_3=\frac{1}{2}\\
        \lambda_2c_2^2+\lambda_3c_3^2=\frac{1}{3}\\
        \lambda_3c_2a_{32}=\frac{1}{6}
    \end{cases}
\end{equation*}

\subsection{三阶Kutta方法}
取$\lambda_1=1/6,\lambda_2=2/3,\lambda_3=1/6,c_2=1/2,c_3=1,a_{32}=2,a_{31}=-1$
\begin{equation*}
    \begin{cases}
        y_{i+1}=y_i+h\left(\frac{1}{6}k_1+\frac{2}{3}k_2+\frac{1}{6}k_3\right)\\
        k_1=f(x_i,y_i)\\
        k_2=f(x_i+\frac{1}{2}h,y_i+\frac{1}{2}hk_1)\\
        k_3=f(x_i+h,y_i-hk_1+2hk_2)
    \end{cases}
\end{equation*}

\subsection{三阶Hune方法}
取$\lambda_1=1/4,\lambda_2=0,\lambda_3=3/4,c_2=1/3,c_3=2/3,a_{32}=2/3,a_{31}=0$
\begin{equation*}
    \begin{cases}
        y_{i+1}=y_i+h\left(\frac{1}{4}k_1+\frac{3}{4}k_3\right)\\
        k_1=f(x_i,y_i)\\
        k_2=f(x_i+\frac{1}{3}h,y_i+\frac{1}{3}hk_1)\\
        k_3=f(x_i+\frac{2}{3}h,y_i+\frac{2}{3}hk_2)
    \end{cases}
\end{equation*}

\section{四阶Runge-Kutta方法}

\begin{equation*}
    \begin{cases}
        y_{i+1}=y_i+h(\lambda_1k_1+\lambda_2k_2+\lambda_3k_3+\lambda_4k_4)\\
        k_1=f(x_i,y_i)\\
        k_2=f(x_i+c_2h,y_i+c_2hk_1)\\
        k_3=f(x_i+c_3h,y_i+a_{31}hk_1+a_{32}hk_2)\\
        k_4=f(x_i+c_4h,y_i+a_{41}hk_1+a_{42}hk_2+a_{43}hk_3)\\
        c_3=a_{31}+a_{32}, c_4=a_{41}+a_{42}+a_{43}
    \end{cases}
\end{equation*}
其中
\begin{equation*}
    \begin{cases}
        a_{31}+a_{32}=c_3\\
        a_{41}+a_{42}+a_{43}=c_4\\
        \lambda_1+\lambda_2+\lambda_3+\lambda_4=1\\
        \lambda_2c_2+\lambda_3c_3+\lambda_4c_4=\frac{1}{2}\\
        \lambda_2c_2^2+\lambda_3c_3^2+\lambda_4c_4^2=\frac{1}{3}\\
        \lambda_2c_2^3+\lambda_3c_3^3+\lambda_4c_4^3=\frac{1}{4}\\
        \lambda_3c_2a_{32}+\lambda_4(c_2a_{42}+c_3a_{43})=\frac{1}{6}\\
        \lambda_3c_2^2a_{32}+\lambda_4(c_2^2a_{42}+c_3^2a_{43})=\frac{1}{12}\\
        \lambda_3c_2c_3a_{32}+\lambda_4c_4(c_2a_{42}+c_3a_{43})=\frac{1}{8}\\
        \lambda_4c_2a_{32}a_{43}=\frac{1}{24}
    \end{cases}
\end{equation*}

\subsection{四阶经典Runge-Kutta方法}
\begin{equation*}
    \begin{cases}
        y_{i+1}=y_i+\frac{h}{6}(k_1+2k_2+2k_3+k_4)\\
        k_1=f(x_i,y_i)\\
        k_2=f(x_i+\frac{h}{2},y_i+\frac{h}{2}k_1)\\
        k_3=f(x_i+\frac{h}{2},y_i+\frac{h}{2}k_2)\\
        k_4=f(x_i+h,y_i+hk_3)
    \end{cases}
\end{equation*}