\chapter{非线性方程求根}

\section*{学习思路}

\begin{equation}\label{eqn:8.1.1}
    f(x) = 0
\end{equation}

\begin{enumerate}
    \item 根的存在性;
    \item 根的搜索;
    \item 根的精确化。
\end{enumerate}

\subsection{根的存在性}

\begin{definition}
    如果存在$x^*$使得$f(x^*) = 0$,则称$x^*$为方程(\ref{eqn:8.1.1})的根或函数$f(x)$的零点。
\end{definition}

\begin{theorem}
    设函数$f(x)$在区间$[a,b]$上连续,如果
    \begin{equation*}
        f(a) \cdots f(b) < 0
    \end{equation*}
    则方程$f(x) = 0$在$[a,b]$内至少有一实根$x^*$。
\end{theorem}

\begin{definition}[m重根]
    若
    \begin{equation*}
        f(x) = (x-x^*)^mg(x)
    \end{equation*}
    其中,$g(x^*) \neq 0,m$为正整数,则当$m = 1$时,称$x^*$为方程(\ref{eqn:8.1.1})的单根或函数$f(x)$的单零点。
    当$m \geq 2$时,称$x^*$为方程(\ref{eqn:8.1.1})的m重根或函数$f(x)$的m重零点。
\end{definition}

\subsection{根的搜索}

\begin{enumerate}
    \item 图解法
    \item 解析法
    \item 近似方程法
    \item 定步长搜索法$\surd$ 
\end{enumerate}

\begin{definition}[定步长搜索法]
    1. 画出$f(x)$的略图,从而看出曲线与$x$轴交点的位置。
    2.从左端点$x=a$出发,按某个预先选定的步长$h$一步一步地向右跨,每跨一步都检验每步起点$x_0$和终点$x_0+h$的函数值,若$f(x_0) \cdot f(x_0+h) \leq 0$,那么所求的根$x^*$必在$x_0$与$x_0+h$之间,这里可取$x_0$或$x_0+h$作为根的初始近似。
\end{definition}

\begin{example}
    考察方程
    \begin{equation*}
        f(x) = x^3-x-1 = 0
    \end{equation*}
    %这里有个表格不会打
\end{example}

\section{二分法}
%两个图片
$|x_{k+1}-x_k < \varepsilon_1$或$|f(x)| < \varepsilon_2 $
后者不能保证$x$的精度。

\begin{enumerate}
    \item 计算$f(x)$在有界区间$[a,b]$端点处的值$f(a),f(b)$。
    \item 计算区间中点$x_1$及$f(x)$在区间中点处的函数值$f(x_1)$。
    \item 判断若$f(x_1) = 0$,则$x_1$即是根,否则检验:
    \begin{enumerate}
        \item 若$f(x_1)$与$f(a)$异号,则知解位于区间$[a,x_1]$,$b_1 = x_1,a_1 = a$;
        \item 若$f(x_1)$与$f(a)$同号,则知解位于区间$[x_1,b]$,$a_1 = x_1,b_1 = b$;
    \end{enumerate}
        反复执行步骤2、3,便可得到一系列有根区间:
        \begin{equation*}
            [a,b]\supset [a_1,b_1] \supset [a_2,b_2] \supset \cdots \supset [a_n,b_n] \supset \cdots
        \end{equation*}
    \item 当$|b_{k+1}-a_{k+1}| < \varepsilon$时,停止;
        $x_{k+1} = \frac{1}{2}(a_k+b_k)$即为根的近似。
\end{enumerate}

\begin{remark}
    当$n \rightarrow \infty $时,$b_n-a_n \rightarrow 0$,即这些区间必将收缩于一点,也就是方程的根。在实际计算中,只要$[a_n,b_n]$的区间长度小于预定容许误差$\varepsilon$就可以停止搜索,即
    \begin{equation*}
        \frac{b-a}{2^n} < \varepsilon
    \end{equation*}
    然后取其中点$x_n$作为方程的一个根的近似值。
\end{remark}

\begin{example}
    证明方程$e^x+10x-2 = 0$存在唯一的实根$x^* \in (0,1)$。用二分法求出此根,要求误差不超过$0.5 \times 10^{-2}$。
    \begin{solution}
    解:
        记$f(x) = e^x+10x-2$则对任意$x \in R$,
        \begin{equation*}
            f'(x) = e^x+10 > 0
        \end{equation*}
        因而,$f(x)$是严格单调的,$f(x) = 0$最多有一个根,
        又因为$f(0) = -1 < 0,f(1) = e+8 > 0$
        所以,$f(x) = 0$,有唯一实根$x^* \in (0,1)$
        用二分法求解,要使$|x_k-x^*| \leq 0.5 \times 10^{-2}$,只要
        \begin{equation*}
            \frac{1-0}{2^{k+1}} \leq 0.5 \times 10^{-2}
        \end{equation*}
        解得$k \geq \frac{2}{\lg 2} = 6.64$,取$k = 7$。所以只要二等分7次,即可求得满足精度要求的根。计算过程如表8.2.1所示。
        %少了个表8.2.1
        所以,$x^* \approx 0.5 \times (0.0859375+0.09375) \approx 0.09$ 
    \end{solution}
 \end{example}

\begin{extend}[二分法的优缺点]
    优点:\begin{enumerate}
        \item 简单;
        \item 对$f(x)$要求不高(只要连续即可)。
    \end{enumerate}
    缺点:\begin{enumerate}
        \item 无法求复根及偶重根;
        \item 收敛慢。
    \end{enumerate}
\end{extend}

\section{迭代法}

使用迭代法求解非线性代数方程的步骤为:
\begin{enumerate}
    \item 迭代格式的构造;
    \item 迭代格式的收敛性分析;
    \item 迭代格式的收敛速度与误差分析。
\end{enumerate}

\subsection{简单迭代法}

\subsubsection{迭代格式的构造}
\begin{equation*}
    f(x) = 0 \Longleftrightarrow x = \varphi(x)
\end{equation*}
方程(8.3.1)称为不动点方程,$\varphi(x)$称为迭代函数。
满足(8.3.1)式的点称为不动点。这样就将求$f(x)$的零点问题转化为求$\varphi(x)$的不动点问题。
以不动点方程为原型构造迭代格式
\begin{equation*}
    x_{k+1} = \varphi(x_k),(k = 0,1,\cdots)
\end{equation*}
称这种迭代格式为不动点迭代。

\begin{example}
    求代数方程$2x^3-x-1 = 0$的根。
    \begin{solution}
        解:显然,1是方程的一个根,位于区间$[0,1.5]$内。

        迭代格式1:
        将方程等价变换为:$x = 2x^3-1$
        构造迭代格式:$x_{k+1} = 2x^3_k-1$
        取迭代初始值:$x_0 = 0$
        迭代过程:
        \begin{equation*}
            \left\{
                \begin{aligned}x_0 & =0 & \\
                x_1 & =2x_0^3-1=-1\\
                x_2 & =2x_1^3-1=-3\\
                x_3 & =2x_2^3-1=-55\\
                  & \cdots
                \end{aligned}
            \right.
        \end{equation*}
        上式显然迭代发散。

        迭代格式2:
        将方程等价变换为:$x = \sqrt[3]{\frac{x+1}{2}} $
        取初值:$x_0 = 0$
        迭代过程:
        \begin{equation*}
            \left\{
                \begin{aligned}
                    x_0 & =0\\
                    x_1 & =\sqrt[3]{\frac{x_0+1}2}=\sqrt[3]{\frac12}\approx0.7937\\
                    x_2 & =\sqrt[3]{\frac{x_1+1}2}=\sqrt[3]{\frac{1.7937}2}\approx0.9644
                \end{aligned}
            \right.
        \end{equation*}
        以此类推,收敛于1。
    \end{solution}
    \begin{notice}
        同样的方程,不同的迭代公式,不同的收敛性。
    \end{notice}
\end{example}

\begin{remark}
    \begin{enumerate}
        \item 简单迭代法的迭代函数不唯一,迭代函数不同,收敛性不同;
        \item 非线性方程的根不唯一,迭代点列收敛与否不仅与迭代函数有关,还与初始点有关。
    \end{enumerate}
\end{remark}

\subsubsection{迭代过程的收敛性}
\begin{theorem}
    如果$\varphi(x)$满足下列条件
    \begin{enumerate}
        \item 当$x \in [a,b]$,$\varphi(x) \in [a,b]$
        \item 对任意$x \in [a,b]$,存在$0 < L < 1$,使
        \begin{equation*}
            |\varphi'(x)| \leq L < 1
        \end{equation*}
    \end{enumerate}
    则方程$x = \varphi(x)$在$[a,b]$上有唯一的根$x^*$,且对任意初值$x_0 \in [a,b]$,迭代序列$x_{k+1} = \varphi(x_k)(k =0,1,\cdots )$收敛于$x^*$。
    \begin{remark}
        此处$L$可以看成是$|\varphi'(x)|$在区间$[a,b]$内的最大值。
    \end{remark}
\end{theorem}

\subsubsection{迭代法的误差估计}
在定理8.1的条件下,简单迭代法产生的迭代点列有如下误差估计式:
\begin{align}
        |x_k-x^*| \leq \frac{L^k}{1-L}|x_1-x_0|\\
        |x_k-x^*| \leq \frac{1}{1-L}|x_{k+1}-x_k|\\
        |x_k-x^*| \leq \frac{L}{1-L}|x_k-x_{k-1}|
\end{align}

\begin{example}
    求方程$f(x) = x^3+4x^2-10 = 0$在$[1,1.5]$内的根$x^*$。
    \begin{solution}
        解:
        原方程可以等价变形为下列三个迭代格式:
        \begin{equation}\label{eqn:1}
            x_{k+1} = 10+x_k-4x_k^2-x_k^3(k=0,1,2,\cdots )
        \end{equation}
        \begin{equation}\label{eqn:2}
            x_{k+1} \frac{1}{2}\sqrt{10-x_k^3}(k = 0,1,2,\cdots )
        \end{equation}
        \begin{equation}\label{eqn:3}
            x_{k+1} = \sqrt{\frac{10}{x_k+4}}(k = 0,1,2,\cdots )
        \end{equation}

        由迭代格式(\ref{eqn:1})
        取初值$x_0 = 1.25$得:
        \begin{align*}
            x_1 &= 3.046875\\
            x_2 &= -26.04005
        \end{align*}

        \begin{notice}
            结果是发散的。
        \end{notice}

        由迭代格式(\ref{eqn:2})
        取初值$x_0 = 1.25$得:
        \begin{align*}
            x_1 &= \varphi(x_0) = 1.41835,\\
            x_2 &= \varphi(x_1) = 1.33666,\\
            x_3 &= \varphi(x_2) = 1.37948,\\
            x_4 &= \varphi(x_3) = 1.35786,\\
            x_5 &= \varphi(x_4) = 1.36898,\\
            x_6 &= \varphi(x_5) = 1.36331,\\
            x_7 &= \varphi(x_6) = 1.36621,\\
            x_8 &= \varphi(x_7) = 1.36473,\\
            x_9 &= \varphi(x_8) = 1.36549,\\
            x_{10} &= \varphi(x_9) = 1.36510
        \end{align*}
        结果精确到四位有效数字,迭代到$x_{10}$得到收敛结果。
        \begin{notice}
            迭代10步才能得到收敛的结果。
        \end{notice}

        由迭代格式(\ref{eqn:3})
        取初值$x_0 = 1.25$得:
        \begin{align}
            x_1 &= \varphi(x_0) = 1.38013,\\
            x_2 &= \varphi(x_1) = 1.36334,\\
            x_3 &= \varphi(x_2) = 1.36547,\\
            x_4 &= \varphi(x_3) = 1.36512
        \end{align}
        结果精确到四位有效数字,迭代到$x_4$得到收敛结果。
        \begin{notice}
            四步就能得到收敛的结果。
        \end{notice}
    \end{solution}
\end{example}

\begin{extend}
    分析:
    迭代格式(1)的迭代函数为:
    \begin{equation*}
        \varphi(x) = 10+x-4x^2-x^3
    \end{equation*}
    求导得
    \begin{equation*}
        \varphi'(x) = 1-8x-3x^2 
    \end{equation*}
    当$x \in [1,1.5]$时,
    \begin{equation*}
        |\varphi'(x)| = 3x^2+8x-1 \geq 10 > 1
    \end{equation*}
    故迭代格式(1)是发散的。

    迭代格式(2)的迭代函数为:
    \begin{equation*}
        \varphi(x) = \frac{1}{2}\sqrt{10-x^3}
    \end{equation*}
    当$x \in [1,1.5]$时,
    \begin{align*}
        \varphi(x) \in [\varphi(1.5),\varphi(1)] & = [\frac{1}{2}\sqrt{10-{1.5}^3},\frac{1}{2}\sqrt{10-1^3}]\\
        & = [1.287,1.5] \in [1,1.5]
    \end{align*}
    由
        \begin{align*}
            \varphi'(x) &= -\frac{3x^2}{4\sqrt{10-x^3}} < 0\\
            \varphi''(x) &= -\frac{3}{8}x(10-x^3)(10-x^3)^{-\frac{3}{2}} < 0
        \end{align*}
    知,当$x \in [1,1.5]$时,
    \begin{equation*}
        |\varphi'(x)| \leq |\varphi'(1.5)| = \frac{3}{4} \frac{{1.5}^2}{\sqrt{1.-{1.5}^2}} = 0.6556 < 1
    \end{equation*}
    所以迭代格式(2)是收敛的。

    迭代格式(3)的迭代函数为:
    \begin{equation*}
        \varphi(x) = \sqrt{\frac{10}{x+4}}
    \end{equation*}
    当$x \in [1,1.5]$时,
    \begin{align*}
        \varphi(x) = [\varphi(1.5),\varphi(1)] & = \left[\sqrt{\frac{10}{1.5+4}},\sqrt{\frac{10}{1+4}}\right]\\
        & = [1.348,1.414] \subset [1,1.5]
    \end{align*}
    由
        \begin{align*}
            \varphi'(x) &= -\frac{1}{2}\sqrt{10}(x+4)^{-\frac{3}{2}} < 0,\\
            \varphi''(x) &= -\frac{3}{4}\sqrt{10}(x+4)^{\frac{5}{2}} > 0
        \end{align*}
    知,当$x \in [1,1.5]$时,
    \begin{equation*}
        |\varphi'(x)| \leq |\varphi'(1)| = \frac{1}{2}\sqrt{10}(1+4)^{-\frac{3}{2}} = \frac{\sqrt{2}}{10} = 0.1414
    \end{equation*}
    所以迭代格式(3)也是收敛的。

    通过以上算例可以看出,对迭代函数$\varphi(x)$求导,所得到的$|\varphi'(x)|$的霞姐若是大于1,则迭代格式发散;若上界小于1,则收敛;且上界越小收敛速度越快。
\end{extend}

\subsubsection{迭代过程的局部收敛性}
\begin{definition}
    若存在$x^*$的某个邻域N:$|x-x^*| \leq \delta $使迭代过程$x_{k+1} = \varphi(x_k)$对任意初值$x_0 \in N$均收敛,则称迭代过程$x_{k+1} = \varphi(x_k)$在$x^*$邻近具有局部收敛性。
\end{definition}

\begin{theorem}
    设$x^*$为方程$x = \varphi(x)$的根,$\varphi'(x)$在$x^*$的邻近连续且$|\varphi'(x^*)| < 1$,则迭代过程$x^{k+1} = \varphi(x_k)$在$x^*$邻近具有局部收敛性。
\end{theorem}

\subsubsection{迭代过程的收敛速度}
\begin{definition}
    设由某方法确定的序列${x_k}$收敛于方程的根$x^*$,如果存在正实数$p$,使得
    \begin{equation*}
        \lim_{k \to \infty }\frac{|x^*-x_{k+1}|}{|x^*-x_k|} = C(C\text{为非零常数})
    \end{equation*}
    则称序列${x_k}$收敛于$x^*$的收敛速度是$p$阶的,或称该方法具有$p$阶收敛速度。
    当$p = 1$时,称该方法为线性(一次)收敛;
    当$p = 2$时,称该方法为平方(二次)收敛;
    当$1<p<2$或$C = 0,p = 1$时,称该方法为超线性收敛。
\end{definition}

\subsubsection{加速收敛技术}

$L$越小迭代法的收敛速度越快,因此,可以从寻找较小的$L$来改进迭代格式以加快收敛速度。

(1)松弛法
引入待定参数$\lambda \neq -1$,将$x = \varphi(x)$作等价变形为
\begin{equation}
    x = \frac{\lambda}{1+\lambda}x+\frac{1}{1+\lambda}\varphi(x)
\end{equation}
将方程右端记为$\varPsi(x)$,则得到新的迭代格式
\begin{equation*}
    x_{k+1} = \varPsi(x_k)
\end{equation*}
由定理8.1知
\begin{equation*}
    |\varphi'(x)| \leq L < 1
\end{equation*}
为了使新的迭代格式比原来迭代格式收敛得更快,只要满足
\begin{equation*}
    |\varPsi'(x)|<|\varphi'(x)|
\end{equation*}
且$|\varPsi'(x)| = |\frac{1}{1+\lambda}(\lambda+\varphi'(x))|$越小,所获取的$L$就越小,迭代法收敛的就越快,因此我们希望$\varPsi'(x) \rightarrow 0$。
可取$\lambda_k = \varphi'(x_k) \neq -1$,若记$\omega_k = \frac{1}{1+\lambda_k}$,则(8.3.4)式可改写为
\begin{equation*}
    x_{k+1} = (1-\omega_k)x_k+\omega_k\varphi(x_k)
\end{equation*}
$\omega_k$称为松弛因子,这种方法称为松弛法。为使迭代速度加快,需要边计算边调整松弛因子。由于计算松弛因子需要用到微商,在实际应用中不便使用,具有一定局限性。若迭代法是线性收敛的,当计算$\varphi'(x)$不方便时,可以采用埃特金加速公式。

(2)埃特金加速公式
设迭代法是线性收敛的,由定义知
\begin{equation*}
    \lim_{k \to \infty}\frac{x_{k+1}-x^*}{x_k-x^*}=c \neq 0
\end{equation*}
成立,故当$k \to \infty$时有
\begin{equation*}
    \frac{x_{k+1}-x^*}{x_k-x^*} \approx \frac{x_{k+2}-x^*}{x_{k+1}-x^*}
\end{equation*}
由此可得$x^*$的近似值
\begin{equation}
    x^* \approx x_k-\frac{(x_{k+1}-x_k)^2}{x_{k+2}-2x_{k+1}=x_k} = \overline{x_k}
\end{equation}
由此获得比$x_k$,$x_{k+1}$和$x_{k+2}$更好的近似值$\bar{x}_k$。该式称为埃特金加速公式。

(3)Steffensen加速法:
将埃特金加速公式与不动点迭代相结合,可得
\begin{equation*}
    \begin{cases}
        x_{k+1}^{(1)}=\boldsymbol{\varphi}(x_k),\quad x_{k+1}^{(2)}=\boldsymbol{\varphi}(x_{k+1}^{(1)})\\
        x_{_{k+1}}=x_k-\frac{\left(x_{k+1}^{(1)}-x_k\right)^2}{x_{k+1}^{(2)}-2x_{k+1}^{(1)}+x_k},\quad k=0,1,\cdots&
    \end{cases}
\end{equation*}
利用(8.3.6)式构造序列${x_k}$的方法称为Steffensen加速法。
即每进行两次不动点迭代,就执行一次Aitken加速。

\begin{example}
    试用简单迭代法和Steffensen加速法求方程
    \begin{equation*}
        x = e^{-x}
    \end{equation*}
    在$x = 0.5$附近的根,精确至四位有效数。
\end{example}

\begin{solution}
    记$\varphi(x) = e^{-x}$,简单迭代法公式为:\[x_{k+1} = \varphi(x_k),k = 0,1,2,\cdots\]
    计算得
    
    \begin{table}[h!]
        \begin{tabular}{|c|c|c|c|c|c|}
            \hline
            $k$&$x_k$&$k$&$x_k$&$k$&$x_k$\\
            \hline
            0&0.5&7&0.5584380&14&0.5671188\\
            1&0.6065306&8&0.5664094&15&0.5671571\\
            2&0.5452392&9&0.5675596&16&0.5671354\\
            3&0.5797031&10&0.5669072&17&0.5671477\\
            4&0.5600646&11&0.5672772&18&0.5671407\\
            5&0.5711721&12&0.5670673&&\\
            6&0.5648629&13&0.5671863&&\\
            \hline
        \end{tabular}
    \end{table}

    Aitken加速公式:\[x_{k+1}=x_k-\frac{\left(x_{k+1}^{(1)}-x_k\right)^2}{x_{k+1}^{(2)}-2x_{k+1}^{(1)}+x_k}\]
    计算得%%又一个表
    所以,$x^* \approx 0.5671$
\end{solution}

\section{牛顿法}

\subsection{牛顿迭代公式}
考虑非线性方程:
\begin{equation*}
    f(x) = 0
\end{equation*}
取$x_0 \approx x^*$,将$f(x)$在$x_0$做一阶Taylor展开:
\begin{equation*}
    f(x)=f(x_0)+f^{\prime}(x_0)(x-x_0)+\frac{f^{\prime\prime}(\xi)}{2!}(x-x_0)^2
\end{equation*}
$\xi$在$x_0$和$x$之间。

将$(x^*-x_0)^2$看成高阶小量,则有:
\begin{equation*}
    0 = f(x^*) \approx f(x_0)+f'(x_0)(x^*-x_0) \Rightarrow x^* \approx x_0-\frac{f(x_0)}{f'(x_0)}
\end{equation*}
记
\[x_1 = x_0-\frac{f(x_0)}{f'(x_0)}\]
再将$f(x)$在$x_1$做一阶Taylor展开,并忽略二阶项得
\[0=f(x)\approx f(x_1)+f^{\prime}(x_1)(x-x_1)\Longrightarrow x^*\approx x_2=x_1-\frac{f(x_1)}{f^{\prime}(x_1)}\]
依此类推,可构造迭代公式
\begin{equation}
    x_{k+1}=x_k-\frac{f(x_k)}{f'(x_k)}
\end{equation}
公式(6.4.1)称为牛顿迭代公式。

只要$f \in C^1$,且每步迭代都有$f'(x_k) \neq 0$,且$\lim_{k\to\infty}x_k=x^*$,则
\[x^{*}\underset{}{\operatorname*{=}}\lim\limits_{k\to\infty}x_{k+1}=\underset{k\to\infty}{\operatorname*{\lim}}\left[x_{k}-\frac{f(x_k)}{f'(x_k)}\right]=x^{*}-\frac{f(x*)}{f'(x*)}\]
从而
\[f(x^*) = 0\]
即$x^*$就是$f(x)$的根。
故只要牛顿迭代格式收敛,则必收敛到$f(x)$的根。

\subsection{牛顿法的几何意义}

\subsection{牛顿法的收敛性}

\subsubsection{牛顿法收敛的条件}
\begin{theorem}
    设$f(x)$在$[a,b]$上二阶连续可导且满足下列条件。
    \begin{enumerate}
        \item \[f(a) \cdot f(b) < 0;\]
        \item \[f'(x) \neq 0;\]
        \item $f''(x)$在区间$[a,b]$上不变号;
        \item 取$x_0 \in[a,b]$,使得$f''(x)\cdot f(x_0)>0$
    \end{enumerate}
    则由(6.4.1)确定的牛顿迭代序列${x_k}$二阶收敛于$f(x)$在$[a,b]$上的唯一单根$x^*$。
\end{theorem}

\begin{remark}
    Newton法的收敛性依赖于$x_0$的选取。
\end{remark}

\begin{notice}
    Newton法是局部收敛的算法!且是条件最为严格的。
\end{notice}

\subsubsection{迭代法收敛阶的判别}
\begin{theorem}
    如果$\varphi(x)$在$x^*$附近的某个邻域内有$p(p \geq 1)$阶连续导数,且
    \begin{align*}
        &\varphi^{(j)}(x^*)=0 \quad(j=1,2,\ldots,p-1)\\
        &\varphi^{(p)}(x^*) \neq 0
    \end{align*}
    则迭代格式$x_{k+1} = \varphi(x_k)$在$x^*$附近是$p$阶局部收敛的,且有
    \begin{equation*}
        \lim_{k\to\infty}\frac{(x_{k+1}-x^*)}{(x_k-x^*)^p}=\frac{\varphi^{(p)}(x^*)}{p!}
    \end{equation*}
\end{theorem}

\subsubsection{牛顿迭代法的局部收敛性和收敛速度}
若$x^*$是$f(x^*)$的一个单根,即$f(x^*) = 0,f'(x) \neq 0$,且$f$在包含$x^*$的一个区间二阶连续可导,则Newton迭代法至少二阶收敛,即$\varphi'(x^*) = 0$。且
\begin{align*}
    \lim_{k\to\infty}\frac{x_{k+1}-x^*}{\left(x_k-x^*\right)^2}=\frac{f^{\prime\prime}(x^*)}{2f^{\prime}(x^*)}
    \varphi^{\prime}(x)=\frac{f(x)f^{\prime\prime}(x)}{\left[f^{\prime}(x)\right]^2}
\end{align*}
其中,$\varphi(x) = x-\frac{f(x)}{f'(x)}$为牛顿迭代函数。
值得注意的是,当$f(x)$充分光滑且$x^*$是$f(x) = 0$的重根时,牛顿法在$x^*$的附近是线性收敛的。即
\begin{equation*}
    \varphi'(x^*) \neq 0
\end{equation*}

\begin{example}
    用牛顿法求方程的根:$xe^x-1 = 0$
    \begin{solution}
        解:
        由牛顿迭代公式$x_{k+1}=x_k-\frac{f(x_k)}{f'(x_k)}$得:
        \begin{equation*}
            x_{k+1}=x_k-\frac{x_k-e^{-x_k}}{1+x_k}
        \end{equation*}
        取初值$x_0 = 0.5$
        %%计算结果又是一个表
    \end{solution}
\end{example}

\begin{example}
    Leonardo与1225年研究了方程
    \begin{equation*}
        x^3+2x^2+10x-20 = 0
    \end{equation*}
    并得出该方程的一个近似根$x = 1.368808107$。试用牛顿法求出此结果。
    \begin{solution}
        解:
        记\[f(x)=x^3+2x^2+10x-20\]
        则\[f'(x)=3x^2+4x+10=3(x+\frac{2}{3})^2+\frac{26}{3}\]
        当$x \in R$时,$f'(x) > 0$即$f(x)$为单调函数,
        又\[f(1) = -7 < 0,f(2) = 16 > 0\]
        所以$f(x) = 0$有唯一实根$x^* \in(1,2)$。
        改写:
        \begin{align*}
            f(x) = ((x+2)x+10)x-20\\
            f'(x) = (3x+4)x+10
        \end{align*}
        用牛顿迭代格式:
        \begin{equation*}
            \begin{cases}
                x_{k+1} = x_k-\frac{f(x_k)}{f'(x_k)},k=0,1,2,\cdot\\
                x_0 = 1.5
            \end{cases}
        \end{equation*}
        %%表
        所以,$x^* \approx 1.368808107$
    \end{solution}
    \begin{remark}
        牛顿法只用了3步就得到具有10位有效数字的近似解。
    \end{remark}
\end{example}

\subsubsection{牛顿法的计算步骤}
\begin{enumerate}
    \item 准备:$x_0,f_0 = f(x_0),f'_0 = f'(x_0)$
    \item 迭代:$x_1 = x_0-\frac{f_0}{f'_0},f_1 = f(x_1),f'_1 = f'(x_1)$
    \item 控制:若$|f_1|<\varepsilon_1$或$|\delta|<\varepsilon_2$,停$x^* = x_1$;否则转4步
         \begin{equation*}
            \delta = 
            \begin{cases}
                |x_1-x_0| & \text{if} |x_1|<c\\
                \frac{|x_1-x_0|}{|x_1|} & \text{if} |x_1| \geq c
            \end{cases}
         \end{equation*}
    \item 修改:若迭代次数$>N$,或$f'_1 = 0$,则方法失败,否则继续。
\end{enumerate}

\begin{remark}[牛顿迭代法的优缺点]
    优点:公式简单,使用方便,易于编程,收敛速度快,易于求解;
    缺点:计算量大,每次迭代都要计算函数值与导数值。
\end{remark}


\subsubsection{牛顿法应用举例}
1.求正数平方根
设$C > 0$,求$x = \sqrt{C}$
\begin{solution}
    解:
    令$f(x) = x^2-C$则$f'(x) = 2x$
    于是,牛顿迭代公式为
    \begin{equation*}
        x_{k+1} = x_k-\frac{x_k^2-C}{2x_k} \Rightarrow x_{k+1} = \frac{1}{2}[x_k+\frac{C}{x_k}]
    \end{equation*}
    收敛性分析:
    \begin{align*}
        x_{k+1}-\sqrt{c} & = \frac{1}{2}[x_k+\frac{c}{x_k}]-\sqrt{c}\\
        & = \frac{1}{2x_k}[x^2_k-2x_k\sqrt{c}+c]\\
        & = \frac{1}{2x_k}(x_k-\sqrt{c})^2
    \end{align*}
    同理可得:
    \begin{equation*}
        x_{k+1}+\sqrt{c} = \frac{1}{2}[x_k+\frac{c}{x_k}] = \frac{1}{2x_k}(x_k+\sqrt{c})^2
    \end{equation*}
    从而有
    \begin{align*}
        &\frac{x_{k+1}-\sqrt{c}}{x_{k+1}+\sqrt{c}}
        = \left(\frac{x_{k}-\sqrt{c}}{x_{k}-\sqrt{c}}\right)^2 \\
        =& \left(\frac{x_{k-1}-\sqrt{c}}{x_{k-1}-\sqrt{c}}\right)^{2^2}
        = \left(\frac{x_{0}-\sqrt{c}}{x_{0}-\sqrt{c}}\right)^{2^{k+1}}
    \end{align*}
    要使$\lim_{k \to \infty} = \sqrt{c}$,则需$(\frac{x_{0}-\sqrt{c}}{x_{}-\sqrt{c}})^{2^{k+1}} \to 0,k \to \infty$
    令$r = \frac{x_{0}-\sqrt{c}}{x_{}-\sqrt{c}}$,则$|r| < 1$
    故要使平方根的牛顿迭代格式收敛,只需取初值
    \[x_0 > 0\]
    且当迭代格式收敛时,有
    \begin{equation*}
        \lim_{k \to \infty}\frac{|e_{k+1}|}{|e_k|^2} = \frac{1}{2\sqrt{c}}
    \end{equation*}
    即求正数平方根的牛顿迭代格式二阶收敛。
    \begin{equation*}
        x_{k+1} = \frac{1}{2[x_k+\frac{c}{x_k}]}
    \end{equation*}
\end{solution}

\begin{remark}
    求正数平方根的牛顿法是局部收敛的,要求初值>0。
\end{remark}

\begin{example}
    用牛顿法求$\sqrt{115}$
    \begin{solution}
        取$x_0 = 10,c = 115$迭代结果如下:%表
        所以$\sqrt{115} \approx 10.73805$
    \end{solution}
\end{example}

2.求倒数算法
用牛顿法求非零数$a$的倒数,不用除法。
\begin{solution}
    解:
    考虑方程
    \[\frac{1}{x}-a = 0\]
    令\[f(x) = \frac{1}{x}-a\]
    则牛顿迭代格式为
    \[x_{k+1} = x_k-\frac{\frac{1}{x_k}-a}{-\frac{1}{x_k^2}}\]
    即
    \[x_{k+1} = x_k(2-ax_k)\]

    收敛性分析:
    \begin{align*}
        x_{k+1}-\frac{1}{a} & = x_k(2-ax_k)\\
        & = -a(x_k-\frac{1}{a})^2
    \end{align*}
    将上式等价变形得
    \[ax_{k+1}-1 = -(ax_k-1)^2\]
    即
    \[1-ax_{k+1} = (1-ax_k)^2\]
    令$r_k = 1-ax_k$,有
    \[r_{k+1} = r_k^2 = r_{k-1}^{2^2} = \cdots = r_0^{2^k}\]
    要使$r_k \to 0$,需$|r_0| = |1-ax_0| < 1$,即$0 < x_0 <\frac{2}{a}$
    \begin{remark}
        求倒数算法要求初值$x_0$满足$0 < x_0 <\frac{2}{a}$
    \end{remark}
    此时有
    \[\lim_{k \to \infty}\frac{|e_{k+1}|}{|e_k|^2} = a\]
    即求倒数的牛顿算法二阶收敛。
\end{solution}


\subsection{求m重根的牛顿法-修正牛顿法}

设\begin{align*}
    f(x^*) = f'(x^*) = \cdots = f^{(m-1)}(x^*) = 0\\
    f^{(m)}(x^*) \neq 0,(m \geq 2)
\end{align*}

修正格式一:构造迭代格式
\begin{equation}
    x_{k+1} = x_k-m\frac{f(x_k)}{f'(x_k)}
\end{equation}
则迭代格式(6.4.2)至少2阶收敛。
\begin{remark}
    重数m的确定:
    令
    \begin{equation*}
        \varPsi(x) = \frac{[f'(x)]^2}{[f'(x)]^2-f(x)f''(x)}
    \end{equation*}
    则
    \begin{equation*}
        m = \lim_{x \to x^*}\varPsi(x)
    \end{equation*}
\end{remark}

修正格式二:
令
\[u(x) = \frac{f(x)}{f'(x)}\]
则$x^*$是$u(x)$的单根,即$u(x^*) = 0,u'(x^*) \neq 0$
构造迭代格式
\begin{equation}
    x_{k+1} = x_k-\frac{u(x_k)}{u'(x_k)}
\end{equation}
则迭代格式(6.4.3)至少2阶收敛。


\subsection{牛顿法的变形}

\subsubsection{牛顿下山法的基本思想}
由于Newton迭代法的收敛性依赖于初值$x_0$的选取,如果$x_0$离方程的根$x^*$较远,则Newton迭代法可能发散。
为了防止迭代发散,可以将Newton迭代法与下山法结合起来,放宽初值$x_0$的选取范围,为此,将(6.4.1)式修改为:
\begin{equation*}
    x_{k+1} = x_k-\lambda\frac{f(x_k)}{f'(x_k)},(k=0,1,2,\cdots)
\end{equation*}
其中,$0 < \lambda \leq 1$称为下山因子,选择下山因子时,希望$f(x_k)$满足下山法具有的单调性,即
\begin{equation*}
    |f(x_{k+1})| < |f(x_k)|,(k = 0,1,2,\cdots)
\end{equation*}
这种算法称为Newton下山法。

在实际应用中,可选择$\lambda = \lambda_k,0<\lambda_k \leq 1$

\subsubsection{牛顿下山法的计算步骤}
\begin{enumerate}
    \item 选取初始近似值$x_0$;
    \item 取下山因子$\lambda = 1$;
    \item 计算$x_{k+1} = x_k -\lambda\frac{f(x_k)}{f'(x_k)}$
    \item 计算$f(x_{k+1})$,并比较$|f(x_{k+1})|$与$|f(x_k)|$的大小,分一下二种情况
    
    \begin{enumerate}
        \item 若$|f(x_{k+1})| < |f(x_k)|$,则当$|x_{k+1}-x_k| < \varepsilon_2$时,取$x^* \approx x_{k+1}$,计算过程结束;当$\abs{x_{k+1}-x_k} \geq \varepsilon_2$时,则把$x_{k+1}$作为新的近似值,并返回到第2步中。
        \item 若$|f(x_{k+1})| \geq |f(x_k)|$,则当$\lambda \leq \varepsilon_\lambda$时,取$x^* \approx x_{k}$,计算过程结束;否则,若$\lambda \leq \varepsilon_\lambda$,而$|f(x_{k+1})| \geq \varepsilon_1$时,则把$x_{k+1}$加上一个适当选定的小正数,即取$x_{k+1}+\delta$作为新的$x_k$值,并转向第3步重复计算;当$\lambda > \varepsilon_\lambda$且$|f(x_{k+1})| \geq \varepsilon_1$时,则将下山因子缩小一半,取$\frac{\lambda}{2}$代入,并转向第3步重复计算。
    \end{enumerate}
\end{enumerate}

\begin{example}
    求方程$f(x) = x^3-x-1 = 0$的根。$x_0 = 0.6$
    \begin{solution}
        牛顿下山法的计算结果:%表
    \end{solution}
\end{example}